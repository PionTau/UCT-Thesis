%%%%%%%%%%%%%%%%%%%%%%%%%%%%%%%%%%%%%%%%%%%%%%%%%%%%%%%%%%%%%%%%%%%%%%%%%%
%     This is format.tex file needed for the dmathesis.cls file.  You    %
%  have to  put this file in the same directory with your thesis files.  %
%                Written by M. Imran 2001/06/18                          % 
%                 No Copyright for this file                             % 
%                 Save your time and enjoy it                            % 
%                                                                        % 
%%%%%%%%%%%%%%%%%%%%%%%%%%%%%%%%%%%%%%%%%%%%%%%%%%%%%%%%%%%%%%%%%%%%%%%%%%
%%%%%  Put packages you want to use here and 'fancyhdr' is required   %%%%
%%%%%%%%%%%%%%%%%%%%%%%%%%%%%%%%%%%%%%%%%%%%%%%%%%%%%%%%%%%%%%%%%%%%%%%%%%
\usepackage{fancyhdr}
\usepackage{epsfig}
\usepackage{cite}
\usepackage{graphicx}
\usepackage{amsmath}
\usepackage{theorem}
\usepackage{amssymb}
\usepackage{latexsym}
\usepackage{epic}
\usepackage{slashed}

\usepackage{hyperref}
\hypersetup{
	colorlinks=true,
	linkcolor=blue,
	citecolor=red,
	linktocpage,
}
%%%%%%%%%%%%%%%%%%%%%%%%%%%%%%%%%%%%%%%%%%%%%%%%%%%%%%%%%%%%%%%%%%%%%%%%%%
%%%%%                 Set line spacing = 1.5 here                   %%%%%%
%%%%%%%%%%%%%%%%%%%%%%%%%%%%%%%%%%%%%%%%%%%%%%%%%%%%%%%%%%%%%%%%%%%%%%%%%%
\renewcommand{\baselinestretch}{1.5}
%%%%%%%%%%%%%%%%%%%%%%%%%%%%%%%%%%%%%%%%%%%%%%%%%%%%%%%%%%%%%%%%%%%%%%%%%%
%%%%%                      Your fancy heading                       %%%%%%
%%%%% For the final copy you need to remove '\bfseries\today' below %%%%%%
%%%%%%%%%%%%%%%%%%%%%%%%%%%%%%%%%%%%%%%%%%%%%%%%%%%%%%%%%%%%%%%%%%%%%%%%%%
\pagestyle{fancy}
\renewcommand{\chaptermark}[1]{\markright{\chaptername\ \thechapter.\ #1}}
\renewcommand{\sectionmark}[1]{\markright{\thesection.\ #1}{}}
\lhead[\fancyplain{}{}]%
      {\fancyplain{}{\bfseries\rightmark}}
\chead[\fancyplain{}{}]%
      {\fancyplain{}{}}
\rhead[\fancyplain{}{}]%
      {\fancyplain{}{\bfseries\thepage}}
\lfoot[\fancyplain{}{}]%
      {\fancyplain{}{}}
\cfoot[\fancyplain{}{}]%
      {\fancyplain{}{}}
\rfoot[\fancyplain{}{}]%
      {\fancyplain{}{}}
%%%%%%%%%%%%%%%%%%%%%%%%%%%%%%%%%%%%%%%%%%%%%%%%%%%%%%%%%%%%%%%%%%%%%%%%%%
%%%%%%%%%%%%Here you set the space between the main text%%%%%%%%%%%%%%%%%%
%%%%%%%%%%%%%%%%%%%and the start of the footnote%%%%%%%%%%%%%%%%%%%%%%%%%%
%%%%%%%%%%%%%%%%%%%%%%%%%%%%%%%%%%%%%%%%%%%%%%%%%%%%%%%%%%%%%%%%%%%%%%%%%%
\addtolength{\skip\footins}{5mm}
%%%%%%%%%%%%%%%%%%%%%%%%%%%%%%%%%%%%%%%%%%%%%%%%%%%%%%%%%%%%%%%%%%%%%%%%%%
%%%%%      Define new counter so you can have the equation           %%%%%
%%%%%    number 4.2.1a for example, this a gift from J.F.Blowey      %%%%%
%%%%%%%%%%%%%%%%%%%%%%%%%%%%%%%%%%%%%%%%%%%%%%%%%%%%%%%%%%%%%%%%%%%%%%%%%%
\newcounter{ind}
\def\eqlabon{
\setcounter{ind}{\value{equation}}\addtocounter{ind}{1}
\setcounter{equation}{0}
\renewcommand{\theequation}{\arabic{chapter}%
         .\arabic{section}.\arabic{ind}\alph{equation}}}
\def\eqlaboff{
\renewcommand{\theequation}{\arabic{chapter}%
         .\arabic{section}.\arabic{equation}}
\setcounter{equation}{\value{ind}}}
%%%%%%%%%%%%%%%%%%%%%%%%%%%%%%%%%%%%%%%%%%%%%%%%%%%%%%%%%%%%%%%%%%%%%%%%%%
%%%%%%%%%%%%           New theorem you want to use              %%%%%%%%%%
%%%%%%%%%%%%%%%%%%%%%%%%%%%%%%%%%%%%%%%%%%%%%%%%%%%%%%%%%%%%%%%%%%%%%%%%%%
{\theorembodyfont{\rmfamily}\newtheorem{Pro}{{\textbf Proposition}}[section]}
{\theorembodyfont{\rmfamily}\newtheorem{The}{{\textbf Theorem}}[section]}
{\theorembodyfont{\rmfamily}\newtheorem{Def}[The]{{\textbf Definition}}}
{\theorembodyfont{\rmfamily}\newtheorem{Cor}[The]{{\textbf Corollary}}}
{\theorembodyfont{\rmfamily}\newtheorem{Lem}[The]{{\textbf Lemma}}}
{\theorembodyfont{\rmfamily}\newtheorem{Exp}{{\textbf Example}}[section]}
\def\remark{\textbf{Remark}:}
\def\remarks{\textbf{Remarks}:}
\def\bproof{\textbf{Proof}: }
\def\eproof{\hfill$\Box$}
%%%%%%%%%%%%%%%%%%%%%%%%%%%%%%%%%%%%%%%%%%%%%%%%%%%%%%%%%%%%%%%%%%%%%%%%%%
%%%%%%%    Bold font in math mode, a gift from J.F.Blowey       %%%%%%%%%%
%%%%%%%%%%%%%%%%%%%%%%%%%%%%%%%%%%%%%%%%%%%%%%%%%%%%%%%%%%%%%%%%%%%%%%%%%%
\def\bv#1{\mbox{\boldmath$#1$}}
%%%%%%%%%%%%%%%%%%%%%%%%%%%%%%%%%%%%%%%%%%%%%%%%%%%%%%%%%%%%%%%%%%%%%%%%%%
%%%%%%%        New symbol which is not defined in Latex         %%%%%%%%%%
%%%%%%%                 a gift from J.F.Blowey                  %%%%%%%%%%
%%%%%%%%%%%%%%%%%%%%%%%%%%%%%%%%%%%%%%%%%%%%%%%%%%%%%%%%%%%%%%%%%%%%%%%%%%
% The Mean INTegral
% to be used in displaystyle
\def\mint{\textstyle\mints\displaystyle}
% to be used in textstyle
\def\mints{\int\!\!\!\!\!\!{\rm-}\ }
%
% The Mean SUM
% to be used in displaystyle
\def\msum{\textstyle\msums\displaystyle}
% to be used in textstyle
\def\msums{\sum\!\!\!\!\!\!\!{\rm-}\ }
%%%%%%%%%%%%%%%%%%%%%%%%%%%%%%%%%%%%%%%%%%%%%%%%%%%%%%%%%%%%%%%%%%%%%%%%%%
%%%%%%%%%%            Define your short cut here              %%%%%%%%%%%%
%%%%%%%%%%%%%%%%%%%%%%%%%%%%%%%%%%%%%%%%%%%%%%%%%%%%%%%%%%%%%%%%%%%%%%%%%%
\def\poincare{Poincar\'e }
\def\holder{H\"older }


%%%%%%%%%%%%%%%%%%%%%%%%%%%%%%%%%%%%%%%%%%%%%%%%%%%%%%%%%%%%%%%%%%%%%%%%%%
%%%%%%%%%%      MHV Techniques (Tanjona Radonirina)           %%%%%%%%%%%%
%%%%%%%%%%%%%%%%%%%%%%%%%%%%%%%%%%%%%%%%%%%%%%%%%%%%%%%%%%%%%%%%%%%%%%%%%%
%*************************MHV TECHNIQUES PACKAGES AND SETTINGS*******************************************

\usepackage{ytableau}

\newcommand\encircle[1]{%
	\tikz[baseline=(X.base)] 
	\node (X) [draw, shape=circle, inner sep=0] {\strut #1};}

%Angle and Square brakets
\newcommand{\abs}[1]{\lvert#1\rvert}
\newcommand{\Abs}[1]{\left\lvert#1\right\rvert}
\newcommand{\braket}[1]{\langle#1\rangle}
\newcommand{\soft}[3]{\frac{\braket{#1 #2}}{\braket{#1 #3} \braket{#3 #2}}}
\newcommand{\qsoft}[3]{\frac{[#1 #2]}{[#1 #3] [#3 #2]}}
\newcommand{\ssoft}[2]{S_{#1 , #2}^{\sigma_3}}
\newcommand{\xsoft}[2]{S_{#1 , #2}^{\sigma_4}}
\newcommand{\ysoft}[2]{S_{#1 , #2}^{\sigma_5}}

%For the soft factor
\newcommand{\ggsoft}[3]{\begin{tikzpicture} [scale=0.5]
	\draw [thick, fill=lightgray] (0,0) circle [radius=0.5];
	\draw [gluon, thick] (-0.353553391,0.353553391)--(-1.5,1.45);
	\draw [gluon, thick] (0.353553391,0.353553391)--(1.5,1.45);
	\draw [gluon, thick, color=blue] (0,0.5)--(0,1.85);
	\node at (-1.77,1.8) {#1};
	\node at (0,2.1) {\textcolor{blue}{#2}};
	\node at (1.77,1.8) {#3};
	\end{tikzpicture}}
\newcommand{\ggsoftblue}[3]{\begin{tikzpicture} [scale=0.5]
	\draw [thick, fill=lightgray] (0,0) circle [radius=0.5];
	\draw [gluon, thick, color=blue] (-0.353553391,0.353553391)--(-1.5,1.45);
	\draw [gluon, thick] (0.353553391,0.353553391)--(1.5,1.45);
	\draw [gluon, thick, color=blue] (0,0.5)--(0,1.85);
	\node at (-1.77,1.8) {\textcolor{blue}{#1}};
	\node at (0,2.1) {\textcolor{blue}{#2}};
	\node at (1.77,1.8) {#3};
	\end{tikzpicture}}
\newcommand{\pgsoft}[3]{\begin{tikzpicture} [scale=0.5]
	\draw [thick, fill=lightgray] (0,0) circle [radius=0.5];
	\draw [thick] (-0.353553391,0.353553391)--(-1.5,1.45);
	\draw [gluon, thick] (0.353553391,0.353553391)--(1.5,1.45);
	\draw [gluon, thick, color=blue] (0,0.5)--(0,1.85);
	\node at (-1.77,1.8) {#1};
	\node at (0,2.1) {\textcolor{blue}{#2}};
	\node at (1.77,1.8) {#3};
	\end{tikzpicture}}
\newcommand{\gpsoft}[3]{\begin{tikzpicture} [scale=0.5]
	\draw [thick, fill=lightgray] (0,0) circle [radius=0.5];
	\draw [gluon, thick] (-0.353553391,0.353553391)--(-1.5,1.45);
	\draw [thick] (0.353553391,0.353553391)--(1.5,1.45);
	\draw [gluon, thick, color=blue] (0,0.5)--(0,1.85);
	\node at (-1.77,1.8) {#1};
	\node at (0,2.1) {\textcolor{blue}{#2}};
	\node at (1.77,1.8) {#3};
	\end{tikzpicture}}
\newcommand{\gpsoftblue}[3]{\begin{tikzpicture} [scale=0.5]
	\draw [thick, fill=lightgray] (0,0) circle [radius=0.5];
	\draw [gluon, thick, color=blue] (-0.353553391,0.353553391)--(-1.5,1.45);
	\draw [thick] (0.353553391,0.353553391)--(1.5,1.45);
	\draw [gluon, thick, color=blue] (0,0.5)--(0,1.85);
	\node at (-1.77,1.8) {\textcolor{blue}{#1}};
	\node at (0,2.1) {\textcolor{blue}{#2}};
	\node at (1.77,1.8) {#3};
	\end{tikzpicture}}
\newcommand{\ppsoft}[3]{\begin{tikzpicture} [scale=0.5]
	\draw [thick, fill=lightgray] (0,0) circle [radius=0.5];
	\draw [thick] (-0.353553391,0.353553391)--(-1.5,1.45);
	\draw [thick] (0.353553391,0.353553391)--(1.5,1.45);
	\draw [gluon, thick, color=blue] (0,0.5)--(0,1.85);
	\node at (-1.77,1.8) {#1};
	\node at (0,2.1) {\textcolor{blue}{#2}};
	\node at (1.77,1.8) {#3};
	\end{tikzpicture}}

%TikZ diagrams
\usepackage{tikz}
\usetikzlibrary{patterns}
\usetikzlibrary{arrows,shapes}
\usetikzlibrary{trees}
\usetikzlibrary{matrix,arrows} 				
\usetikzlibrary{positioning}				
\usetikzlibrary{calc,through}			
\usetikzlibrary{decorations.pathreplacing} 
\usepackage{pgffor}				

\usetikzlibrary{decorations.pathmorphing}
\usetikzlibrary{decorations.markings}
\tikzset{
	vector/.style={decorate, decoration={snake,amplitude=3pt}, draw},
	gluon/.style={decorate, decoration={coil,amplitude=2.5pt},draw}, 
	provector/.style={decorate, decoration={snake,amplitude=2.5pt}, draw},
	antivector/.style={decorate, decoration={snake,amplitude=-2.5pt}, draw},
	fermion/.style={draw=black, postaction={decorate},
		decoration={markings,mark=at position .55 with {\arrow[draw=black,thick]{>}}}},
	fermionbar/.style={draw=black, postaction={decorate},
		decoration={markings,mark=at position .55 with {\arrow[draw=black,thick]{<}}}},
	fermionnoarrow/.style={draw=black},
	gluon/.style={decorate, draw=black,
		decoration={coil,amplitude=3.3pt, segment length=3.5pt}},
	scalar/.style={dashed,draw=black, postaction={decorate},
		decoration={markings,mark=at position .55 with {\arrow[draw=black]{>}}}},
	scalarbar/.style={dashed,draw=black, postaction={decorate},
		decoration={markings,mark=at position .55 with {\arrow[draw=black]{<}}}},
	scalarnoarrow/.style={dashed,draw=black},
	electron/.style={draw=black, postaction={decorate},
		decoration={markings,mark=at position .55 with {\arrow[draw=black]{>}}}},
	bigvector/.style={decorate, decoration={snake,amplitude=4pt}, draw},
}

\tikzstyle{block} = [draw, rectangle, minimum height=3em, minimum width=6em]
%************************************************************************************************************


















